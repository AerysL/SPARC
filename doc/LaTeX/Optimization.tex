%%%%%%%%%%%%%%%%%%%%%%%%%%%%%%%%%%%%%%%%%%%%%%%%%%%%%%%%%%%%%%%%%%%%%%%%%%%%%%%%%%%%%%%%%%%%%
\begin{frame}[allowframebreaks,c]{} \label{Structural relaxation}

\begin{center}
\Huge \textbf{Structural relaxation}
\end{center}

\end{frame}
%%%%%%%%%%%%%%%%%%%%%%%%%%%%%%%%%%%%%%%%%%%%%%%%%%%%%%%%%%%%%%%%%%%%%%%%%%%%%%%%%%%%%%%%%%%%%





%%%%%%%%%%%%%%%%%%%%%%%%%%%%%%%%%%%%%%%%%%%%%%%%%%%%%%%%%%%%%%%%%%%%%%%%%%%%%%%%%%%%%%%%%%%%%
\begin{frame}[allowframebreaks]{\texttt{RELAX\_FLAG}} \label{RELAX_FLAG}
\vspace*{-12pt}
\begin{columns}
\column{0.4\linewidth}
\begin{block}{Type}
Integer
\end{block}

\begin{block}{Default}
0
\end{block}

\column{0.4\linewidth}
\begin{block}{Unit}
No unit
\end{block}

\begin{block}{Example}
\texttt{RELAX\_FLAG}: 1
\end{block}
\end{columns}

\begin{block}{Description}
Flag for performing structural relaxation. $0$ means no structural relaxation. $1$ represents relaxation of atom positions. $2$ represents optimization of volume with the fractional coordinates of the atoms fixed. $3$ represents full optimization of the cell i.e., both atoms and cell volume are relaxed
\end{block}

\begin{block}{Remark}
This flag should not be specified if \hyperlink{MD_FLAG}{\texttt{MD\_FLAG}} is set to $1$. 
\end{block}

\end{frame}
%%%%%%%%%%%%%%%%%%%%%%%%%%%%%%%%%%%%%%%%%%%%%%%%%%%%%%%%%%%%%%%%%%%%%%%%%%%%%%%%%%%%%%%%%%%%%


%%%%%%%%%%%%%%%%%%%%%%%%%%%%%%%%%%%%%%%%%%%%%%%%%%%%%%%%%%%%%%%%%%%%%%%%%%%%%%%%%%%%%%%%%%%%%
\begin{frame}[allowframebreaks]{\texttt{RELAX\_METHOD}} \label{RELAX_METHOD}
\vspace*{-12pt}
\begin{columns}
\column{0.4\linewidth}
\begin{block}{Type}
String
\end{block}

\begin{block}{Default}
LBFGS
\end{block}

\column{0.4\linewidth}
\begin{block}{Unit}
No unit
\end{block}

\begin{block}{Example}
\texttt{RELAX\_METHOD}: NLCG
\end{block}
\end{columns}

\begin{block}{Description}
Specifies the algorithm for structural relaxation. The choices are `LBFGS' (limited-memory BFGS), `NLCG' (Non-linear conjugate gradient), and `FIRE' (Fast inertial relaxation engine). 
\end{block}

\begin{block}{Remark}
LBFGS is typically the best choice.
\end{block}

\end{frame}
%%%%%%%%%%%%%%%%%%%%%%%%%%%%%%%%%%%%%%%%%%%%%%%%%%%%%%%%%%%%%%%%%%%%%%%%%%%%%%%%%%%%%%%%%%%%%




%%%%%%%%%%%%%%%%%%%%%%%%%%%%%%%%%%%%%%%%%%%%%%%%%%%%%%%%%%%%%%%%%%%%%%%%%%%%%%%%%%%%%%%%%%%%%
\begin{frame}[allowframebreaks]{\texttt{RELAX\_NITER}} \label{RELAX_NITER}
\vspace*{-12pt}
\begin{columns}
\column{0.4\linewidth}
\begin{block}{Type}
Integer
\end{block}

\begin{block}{Default}
300
\end{block}

\column{0.4\linewidth}
\begin{block}{Unit}
No unit
\end{block}

\begin{block}{Example}
\texttt{RELAX\_NITER}: 25
\end{block}
\end{columns}

\begin{block}{Description}
Specifies the maximum number of iterations for the structural relaxation (\hyperlink{RELAX_FLAG}{\texttt{RELAX\_FLAG}}).
\end{block}

\begin{block}{Remark}
If \hyperlink{RESTART_FLAG}{\texttt{RESTART\_FLAG}} is set to $1$, then relaxation will restart from the last atomic configuration and run for maximum of \hyperlink{RELAX_NITER}{\texttt{RELAX\_NITER}} iterations. 
\end{block}

\end{frame}
%%%%%%%%%%%%%%%%%%%%%%%%%%%%%%%%%%%%%%%%%%%%%%%%%%%%%%%%%%%%%%%%%%%%%%%%%%%%%%%%%%%%%%%%%%%%%

%%%%%%%%%%%%%%%%%%%%%%%%%%%%%%%%%%%%%%%%%%%%%%%%%%%%%%%%%%%%%%%%%%%%%%%%%%%%%%%%%%%%%%%%%%%%%
\begin{frame}[allowframebreaks]{\texttt{TOL\_RELAX}} \label{TOL_RELAX}
\vspace*{-12pt}
\begin{columns}
\column{0.4\linewidth}
\begin{block}{Type}
Double
\end{block}

\begin{block}{Default}
5e-4
\end{block}

\column{0.4\linewidth}
\begin{block}{Unit}
Ha/Bohr
\end{block}

\begin{block}{Example}
\texttt{TOL\_RELAX}: 1e-3
\end{block}
\end{columns}

\begin{block}{Description}
Specifies the tolerance for termination of the structural relaxation. The tolerance is defined on the maximum force component (in absolute sense) over all atoms. 
\end{block}

%\begin{block}{Remark}
%\end{block}

\end{frame}
%%%%%%%%%%%%%%%%%%%%%%%%%%%%%%%%%%%%%%%%%%%%%%%%%%%%%%%%%%%%%%%%%%%%%%%%%%%%%%%%%%%%%%%%%%%%%


%%%%%%%%%%%%%%%%%%%%%%%%%%%%%%%%%%%%%%%%%%%%%%%%%%%%%%%%%%%%%%%%%%%%%%%%%%%%%%%%%%%%%%%%%%%%%
\begin{frame}[allowframebreaks]{\texttt{TOL\_RELAX\_CELL}} \label{TOL_RELAX_CELL}
\vspace*{-12pt}
\begin{columns}
\column{0.4\linewidth}
\begin{block}{Type}
Double
\end{block}

\begin{block}{Default}
1e-2
\end{block}

\column{0.4\linewidth}
\begin{block}{Unit}
GPa
\end{block}

\begin{block}{Example}
\texttt{TOL\_RELAX}: 1e-3
\end{block}
\end{columns}

\begin{block}{Description}
Specifies the tolerance for termination of the cell relaxation. The tolerance is defined on the maximum principle stress component. 
\end{block}

%\begin{block}{Remark}
%\end{block}

\end{frame}
%%%%%%%%%%%%%%%%%%%%%%%%%%%%%%%%%%%%%%%%%%%%%%%%%%%%%%%%%%%%%%%%%%%%%%%%%%%%%%%%%%%%%%%%%%%%%


%%%%%%%%%%%%%%%%%%%%%%%%%%%%%%%%%%%%%%%%%%%%%%%%%%%%%%%%%%%%%%%%%%%%%%%%%%%%%%%%%%%%%%%%%%%%%
\begin{frame}[allowframebreaks]{\texttt{RELAX\_MAXDILAT}} \label{RELAX_MAXDILAT}
\vspace*{-12pt}
\begin{columns}
\column{0.4\linewidth}
\begin{block}{Type}
Double
\end{block}

\begin{block}{Default}
1.06
\end{block}

\column{0.4\linewidth}
\begin{block}{Unit}
No unit
\end{block}

\begin{block}{Example}
\texttt{RELAX\_MAXDILAT}: 1.4
\end{block}
\end{columns}

\begin{block}{Description}
The maximum scaling of the volume allowed with respect to the initial volume defined by \hyperlink{CELL}{\texttt{CELL}} and \hyperlink{LATVEC}{\texttt{LATVEC}}. This will determine the upper-bound and lower-bound in the bisection method (Brent's method) for the volume optimization.
\end{block}

%\begin{block}{Remark}
%\end{block}

\end{frame}
%%%%%%%%%%%%%%%%%%%%%%%%%%%%%%%%%%%%%%%%%%%%%%%%%%%%%%%%%%%%%%%%%%%%%%%%%%%%%%%%%%%%%%%%%%%%%



%%%%%%%%%%%%%%%%%%%%%%%%%%%%%%%%%%%%%%%%%%%%%%%%%%%%%%%%%%%%%%%%%%%%%%%%%%%%%%%%%%%%%%%%%%%%%
\begin{frame}[allowframebreaks]{\texttt{NLCG\_SIGMA}} \label{NLCG_SIGMA}
\vspace*{-12pt}
\begin{columns}
\column{0.4\linewidth}
\begin{block}{Type}
Double
\end{block}

\begin{block}{Default}
0.5
\end{block}

\column{0.4\linewidth}
\begin{block}{Unit}
No unit
\end{block}

\begin{block}{Example}
\texttt{NLCG\_SIGMA}: 1
\end{block}
\end{columns}

\begin{block}{Description}
Parameter in the secant method used to control the step length in NLCG (\hyperlink{RELAX_METHOD}{\texttt{RELAX\_METHOD}}). 
\end{block}

\begin{block}{Remark}
Default value works well in most cases.
\end{block}

\end{frame}
%%%%%%%%%%%%%%%%%%%%%%%%%%%%%%%%%%%%%%%%%%%%%%%%%%%%%%%%%%%%%%%%%%%%%%%%%%%%%%%%%%%%%%%%%%%%%


%%%%%%%%%%%%%%%%%%%%%%%%%%%%%%%%%%%%%%%%%%%%%%%%%%%%%%%%%%%%%%%%%%%%%%%%%%%%%%%%%%%%%%%%%%%%%
\begin{frame}[allowframebreaks]{\texttt{L\_HISTORY}} \label{L_HISTORY}
\vspace*{-12pt}
\begin{columns}
\column{0.4\linewidth}
\begin{block}{Type}
Integer
\end{block}

\begin{block}{Default}
20
\end{block}

\column{0.4\linewidth}
\begin{block}{Unit}
No unit
\end{block}

\begin{block}{Example}
\texttt{L\_HISTORY}: 15
\end{block}
\end{columns}

\begin{block}{Description}
Size of history in LBFGS (\hyperlink{RELAX_METHOD}{\texttt{RELAX\_METHOD}}).
\end{block}

\begin{block}{Remark}
Default value works well in most cases.
\end{block}

\end{frame}
%%%%%%%%%%%%%%%%%%%%%%%%%%%%%%%%%%%%%%%%%%%%%%%%%%%%%%%%%%%%%%%%%%%%%%%%%%%%%%%%%%%%%%%%%%%%%


%%%%%%%%%%%%%%%%%%%%%%%%%%%%%%%%%%%%%%%%%%%%%%%%%%%%%%%%%%%%%%%%%%%%%%%%%%%%%%%%%%%%%%%%%%%%%
\begin{frame}[allowframebreaks]{\texttt{L\_FINIT\_STP}} \label{L_FINIT_STP}
\vspace*{-12pt}
\begin{columns}
\column{0.4\linewidth}
\begin{block}{Type}
Double
\end{block}

\begin{block}{Default}
5e-3
\end{block}

\column{0.4\linewidth}
\begin{block}{Unit}
Bohr
\end{block}

\begin{block}{Example}
\texttt{L\_FINIT\_STP}: 0.01
\end{block}
\end{columns}

\begin{block}{Description}
Step length for line optimizer in LBFGS (\hyperlink{RELAX_METHOD}{\texttt{RELAX\_METHOD}}).
\end{block}

\begin{block}{Remark}
Default value works well in most cases.
\end{block}

\end{frame}
%%%%%%%%%%%%%%%%%%%%%%%%%%%%%%%%%%%%%%%%%%%%%%%%%%%%%%%%%%%%%%%%%%%%%%%%%%%%%%%%%%%%%%%%%%%%%


%%%%%%%%%%%%%%%%%%%%%%%%%%%%%%%%%%%%%%%%%%%%%%%%%%%%%%%%%%%%%%%%%%%%%%%%%%%%%%%%%%%%%%%%%%%%%
\begin{frame}[allowframebreaks]{\texttt{L\_MAXMOV}} \label{L_MAXMOV}
\vspace*{-12pt}
\begin{columns}
\column{0.4\linewidth}
\begin{block}{Type}
Double
\end{block}

\begin{block}{Default}
0.2
\end{block}

\column{0.4\linewidth}
\begin{block}{Unit}
Bohr
\end{block}

\begin{block}{Example}
\texttt{L\_MAXMOV}: 1.0
\end{block}
\end{columns}

\begin{block}{Description}
The maximum allowed step size in LBFGS (\hyperlink{RELAX_METHOD}{\texttt{RELAX\_METHOD}}).
\end{block}

\begin{block}{Remark}
Default value works well in most cases.
\end{block}

\end{frame}
%%%%%%%%%%%%%%%%%%%%%%%%%%%%%%%%%%%%%%%%%%%%%%%%%%%%%%%%%%%%%%%%%%%%%%%%%%%%%%%%%%%%%%%%%%%%%



%%%%%%%%%%%%%%%%%%%%%%%%%%%%%%%%%%%%%%%%%%%%%%%%%%%%%%%%%%%%%%%%%%%%%%%%%%%%%%%%%%%%%%%%%%%%%
\begin{frame}[allowframebreaks]{\texttt{L\_AUTOSCALE}} \label{L_AUTOSCALE}
\vspace*{-12pt}
\begin{columns}
\column{0.4\linewidth}
\begin{block}{Type}
Integer
\end{block}

\begin{block}{Default}
1
\end{block}

\column{0.4\linewidth}
\begin{block}{Unit}
No unit
\end{block}

\begin{block}{Example}
\texttt{L\_AUTOSCALE}: 0
\end{block}
\end{columns}

\begin{block}{Description}
Flag for automatically determining the inverse curvature that is used to determine the direction for next iteration in LBFGS (\hyperlink{RELAX_METHOD}{\texttt{RELAX\_METHOD}}).
\end{block}

\begin{block}{Remark}
Default works well in most cases.
\end{block}

\end{frame}
%%%%%%%%%%%%%%%%%%%%%%%%%%%%%%%%%%%%%%%%%%%%%%%%%%%%%%%%%%%%%%%%%%%%%%%%%%%%%%%%%%%%%%%%%%%%%

%%%%%%%%%%%%%%%%%%%%%%%%%%%%%%%%%%%%%%%%%%%%%%%%%%%%%%%%%%%%%%%%%%%%%%%%%%%%%%%%%%%%%%%%%%%%%
\begin{frame}[allowframebreaks]{\texttt{L\_LINEOPT}} \label{L_LINEOPT}
\vspace*{-12pt}
\begin{columns}
\column{0.4\linewidth}
\begin{block}{Type}
Integer
\end{block}

\begin{block}{Default}
1
\end{block}

\column{0.4\linewidth}
\begin{block}{Unit}
No unit
\end{block}

\begin{block}{Example}
\texttt{L\_LINEOPT}: 0
\end{block}
\end{columns}

\begin{block}{Description}
Flag for atomic force based line minimization in LBFGS (\hyperlink{RELAX_METHOD}{\texttt{RELAX\_METHOD}}). 
\end{block}

\begin{block}{Remark}
Required only if \hyperlink{L_AUTOSCALE}{\texttt{L\_AUTOSCALE}} is $0$.
\end{block}

\end{frame}
%%%%%%%%%%%%%%%%%%%%%%%%%%%%%%%%%%%%%%%%%%%%%%%%%%%%%%%%%%%%%%%%%%%%%%%%%%%%%%%%%%%%%%%%%%%%%


%%%%%%%%%%%%%%%%%%%%%%%%%%%%%%%%%%%%%%%%%%%%%%%%%%%%%%%%%%%%%%%%%%%%%%%%%%%%%%%%%%%%%%%%%%%%%
\begin{frame}[allowframebreaks]{\texttt{L\_ICURV}} \label{L_ICURV}
\vspace*{-12pt}
\begin{columns}
\column{0.4\linewidth}
\begin{block}{Type}
Double
\end{block}

\begin{block}{Default}
1.0
\end{block}

\column{0.4\linewidth}
\begin{block}{Unit}
No unit
\end{block}

\begin{block}{Example}
\texttt{L\_ICURV}: 0.1
\end{block}
\end{columns}

\begin{block}{Description}
Initial inverse curvature, used to construct the inverse Hessian matrix in LBFGS (\hyperlink{RELAX_METHOD}{\texttt{RELAX\_METHOD}}). 
\end{block}

\begin{block}{Remark}
Needed only if  \hyperlink{L_AUTOSCALE}{\texttt{L\_AUTOSCALE}} is $0$. Default value works well in most cases.
\end{block}

\end{frame}
%%%%%%%%%%%%%%%%%%%%%%%%%%%%%%%%%%%%%%%%%%%%%%%%%%%%%%%%%%%%%%%%%%%%%%%%%%%%%%%%%%%%%%%%%%%%%

%%%%%%%%%%%%%%%%%%%%%%%%%%%%%%%%%%%%%%%%%%%%%%%%%%%%%%%%%%%%%%%%%%%%%%%%%%%%%%%%%%%%%%%%%%%%%
\begin{frame}[allowframebreaks]{\texttt{FIRE\_DT}} \label{FIRE_DT}
\vspace*{-12pt}
\begin{columns}
\column{0.4\linewidth}
\begin{block}{Type}
Double
\end{block}

\begin{block}{Default}
1
\end{block}

\column{0.4\linewidth}
\begin{block}{Unit}
Femto second
\end{block}

\begin{block}{Example}
\texttt{FIRE\_DT}: 0.1
\end{block}
\end{columns}

\begin{block}{Description}
Time step used in FIRE (\hyperlink{RELAX_METHOD}{\texttt{RELAX\_METHOD}}).
\end{block}

\begin{block}{Remark}
Default value works well in most cases.
\end{block}

\end{frame}
%%%%%%%%%%%%%%%%%%%%%%%%%%%%%%%%%%%%%%%%%%%%%%%%%%%%%%%%%%%%%%%%%%%%%%%%%%%%%%%%%%%%%%%%%%%%%


%%%%%%%%%%%%%%%%%%%%%%%%%%%%%%%%%%%%%%%%%%%%%%%%%%%%%%%%%%%%%%%%%%%%%%%%%%%%%%%%%%%%%%%%%%%%%
\begin{frame}[allowframebreaks]{\texttt{FIRE\_MASS}} \label{FIRE_MASS}
\vspace*{-12pt}
\begin{columns}
\column{0.4\linewidth}
\begin{block}{Type}
Double
\end{block}

\begin{block}{Default}
1.0
\end{block}

\column{0.4\linewidth}
\begin{block}{Unit}
Atomic mass unit
\end{block}

\begin{block}{Example}
\texttt{FIRE\_MASS}: 2.5
\end{block}
\end{columns}

\begin{block}{Description}
Pseudomass used in FIRE (\hyperlink{RELAX_METHOD}{\texttt{RELAX\_METHOD}}).
\end{block}

\begin{block}{Remark}
Default value works well in most cases.
\end{block}

\end{frame}
%%%%%%%%%%%%%%%%%%%%%%%%%%%%%%%%%%%%%%%%%%%%%%%%%%%%%%%%%%%%%%%%%%%%%%%%%%%%%%%%%%%%%%%%%%%%%



%%%%%%%%%%%%%%%%%%%%%%%%%%%%%%%%%%%%%%%%%%%%%%%%%%%%%%%%%%%%%%%%%%%%%%%%%%%%%%%%%%%%%%%%%%%%%
\begin{frame}[allowframebreaks]{\texttt{FIRE\_MAXMOV}} \label{FIRE_MAXMOV}
\vspace*{-12pt}
\begin{columns}
\column{0.4\linewidth}
\begin{block}{Type}
Double
\end{block}

\begin{block}{Default}
0.2
\end{block}

\column{0.4\linewidth}
\begin{block}{Unit}
Bohr
\end{block}

\begin{block}{Example}
\texttt{FIRE\_MAXMOV}: 1.0
\end{block}
\end{columns}

\begin{block}{Description}
Maximum movement for any atom in FIRE (\hyperlink{RELAX_METHOD}{\texttt{RELAX\_METHOD}}).
\end{block}

\begin{block}{Remark}
Default value works well in most cases.
\end{block}

\end{frame}
%%%%%%%%%%%%%%%%%%%%%%%%%%%%%%%%%%%%%%%%%%%%%%%%%%%%%%%%%%%%%%%%%%%%%%%%%%%%%%%%%%%%%%%%%%%%%


